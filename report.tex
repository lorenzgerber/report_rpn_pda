\documentclass[a4paper,11pt,twoside]{article}
%\documentclass[a4paper,11pt,twoside,se]{article}

\usepackage{UmUStudentReport}
\usepackage{verbatim}   % Multi-line comments using \begin{comment}
\usepackage{courier}    % Nicer fonts are used. (not necessary)
\usepackage{pslatex}    % Also nicer fonts. (not necessary)
\usepackage[pdftex]{graphicx}   % allows including pdf figures
\usepackage{listings}
%\usepackage{lmodern}   % Optional fonts. (not necessary)
%\usepackage{tabularx}
%\usepackage{microtype} % Provides some typographic improvements over default settings
%\usepackage{placeins}  % For aligning images with \FloatBarrier
%\usepackage{booktabs}  % For nice-looking tables
%\usepackage{titlesec}  % More granular control of sections.

% DOCUMENT INFO
% =============
\department{Institution för Datavetenskap}
\coursename{DV2: Algorithms and Problemsolving 7.5 p}
\coursecode{DV169VT16}
\title{OU5 Automaton}
\author{Lorenz Gerber ({\tt{dv15lgr@cs.umu.se}})} 
\date{2016-03-21}
%\revisiondate{2016-01-18}
\instructor{Lena Kallin Westin / Erik Moström / Jonathan Westin}


% DOCUMENT SETTINGS
% =================
\bibliographystyle{plain}
%\bibliographystyle{ieee}
\pagestyle{fancy}
\raggedbottom
\setcounter{secnumdepth}{2}
\setcounter{tocdepth}{2}
%\graphicspath{{images/}}   %Path for images

\usepackage{float}
\floatstyle{ruled}
\newfloat{listing}{thp}{lop}
\floatname{listing}{Listing}


% DEFINES
% =======
%\newcommand{\mycommand}{<latex code>}

% DOCUMENT
% ========
\begin{document}
\lstset{language=C}
\maketitle
\thispagestyle{empty}
\newpage
\tableofcontents
\thispagestyle{empty}
\newpage

\clearpage
\pagenumbering{arabic}

\section{Introduction} 
The subject of this assignment was to design and construct Push Down
Automaton (PDA) as a general datatype and apply it to implement a
specific PDA that can process input according to \textit{reverse
  polish notation} (RPN). 

\subsection{Push Down Automata Implementation}
The lab assignment proposed to use either a representation as table or
as a graph. It was also communicated that the implementation shall be
finite and non-deterministic.

The PDA was implemented as a struct with the fields \textbf{current
  state}, \textbf{current input}, \textbf{stack}, \textbf{table of states}

\section{Program Structure}
The program was implemented using a conventional C structure with a lean
main function that first defines and initializes variables, then calls
for a function that creates and configures the push down
automaton and finally the applying the command line argument to the
configured pda.

\subsection{Own Datatypes}

\subsubsection{pda - Push Down Automaton}
A generic implementation of a push down automaton according to Sipser
\cite[pp 112-125]{sipser2012}.

\begin{itemize}
\item current state
\item current input
\item register
\item input alphabet dlist with function pointers
\item stack
\item stack alphabet, dlist with function pointers
\item states table
\end{itemize}

\begin{itemize}
\item create
\item add input alphabet character
\item add stack alphabet character
\item add state
\item add transition
\item run automaton
\end{itemize}

The datatype \textit{pda} is constructed from a struct. It contains a
table (from course datatypes \cite{datatypes}, constructed from
dynamic list) with \textit{states}, a stack (from course datatypes
\cite{datatypes}, \textit{stack\_1cell}). The \textit{states} table contains
the transitions.

\subsubsection{Alphabets}
The alphabets could be implemented as single chars. Then it is however
difficult to define a more generic group of chars such as number or
operator. To provide this possibility an alphabet character could also
be as a list or an array of unsigned chars. A more elegant way to
solve this issue would be to define a character by a function,
implemented through function pointers. This opens up for very flexible
definition of alphabet characters. For the current implementation it
was decided to define the alphabet by function pointers and functions.  

\subsubsection{States and Transitions}
The representation of \textit{state} and \textit{transition} for
a table based pda can be done in various ways and the distinction
between \textit{state} and \textit{transition} is less clear than in a
graph based model. Here two different ways were considered: Either
states constructed only as a container for transitions, without any 
reference to the alphabet. This would require a more complex
transition datatype. Also the datatype transition becomes less generic
as it fits just for one specified pair of input and stack values.

Alternatively, \textit{states} could also be implemented according to
the example 2.14 in Sipser \cite[p. 114]{sipser2012}: The state is
represented by triple nested table or
an aggregated array where there is a multi column for each letter in
the input alphabet with subgroups as the individual column for each
letter of the stack alphabet. Implementing a representation for this
model could be done with a nested table, an array, where the logic for
accessing the different levels is integrated in the code or a tree
structure. Such a representation has the advantage that it is directly
visible whether a transition for a certain state is already defined or
not as it has a unique location in the data structure. When choosing a
representation with states as mere containers for transitions, a
control mechanism to prevent assignment of duplicate transitions is
needed. 




A state has an numeric \textit{id} and a table with \textit{id's} of possible
transitions to proceed along. 

%\begin{table}
%\caption{\textit{structure of the \textbf{states} datatype}}
\begin{itemize}
\item id
\item description
\item table with allowed transitions
\end{itemize}
%\end{table}

\textit{struct} contains a table constructed from \textit{dynamic
  list}. The table keys are integers with \textit{structs} as data
container. 


\subsubsection{Transition}
A transition needs to know whether it matches the current
state, it needs to define how to modify the current state and the id
of the new state. 
\textit{struct} 
\textit{table} constructed from \textit{dynamic list}. 

pseudo example:
create automata
set input alphabet as list of unsigned chars
set stack alphabet as list of unsigned chars
set 

\section{Discussion}

\section{Conclusions}

\addcontentsline{toc}{section}{\refname}
\bibliography{references}

\end{document}
